\documentclass{report}

\begin{document}

\title{CoreConstraint Documentation \\ \large The theory behind the CoreConstraint solver}
\author{Blake Loring}
\date{\today}

\maketitle

\chapter {Introduction}

CoreConstraint is designed to be a simplistic, integer only, abstract constraint solver. Allowing CoreVM to have a simple interface to a constraint solver which can verify whether a path is satisfiable by looking at path constraints.

\chapter {Constraint Solvers}

A constraint solver is a piece of software that can take a set of constraints (Such as X > 50, Y < 40, X + Y = 60, X > 0, Y > 0) and find out A. if there is some allocation of variables which satisfies them and B. An allocation of variables which satisfies them.

\chapter {Theory}

The CoreConstraint constraint solver only has to be able to be able to verify linear, integer only constraints. In order to do this the constraints problem is modeled as a optimization problem and then solved using the Simplex algorithm.

\chapter {Implementation}

The constraints solver comes in two parts, the simplified interface used to construct a problem definition (Constraints::Problem) and the solver (Simplex::Table, Simplex::Solver).

\chapter {Results}
\chapter {Conclusion}

\end{document}

\documentclass{article}

\title{CoreASM Documentation \\ \large Documentation on the CoreASM compiler}
\author{Blake Loring}
\date{\today}

\usepackage[parfill]{parskip}

\begin{document}
\maketitle

\section* {Instruction}

In order to make building code for the CoreVM virtual machine easier a simple assembler was constructed.

Similarly to the VirtualMachine CoreASM supports an integer only syntax and instructions (no floats).

The assembler itself uses a LL(1) grammar and subsequently has a very simple parser which is able to generate machine code for CoreVM in a single pass. 

\section* {Syntax}

The syntax of the assembly language is very simple, supporting only labels, data declarations and instructions.

\section* {Usage}

\section* {Implementation}
\section* {Examples}
\section* {Grammar}

TODO: Instruction Specifics (When Finalized).

\begin{verbatim}
NUM: [0-9]+
ID: [a-zA-Z][a-zA-Z0-9]*

Program: Block [Program]
Block: Label | Instruction | Data
Label: ID ':'
Data: 'db' NUM
Instruction:
	Add | Sub | Mul | Div | Load | Move | Set | Get | JumpEqual | JumpNotEqual | Interrupt
\end{verbatim}
\end{document}
